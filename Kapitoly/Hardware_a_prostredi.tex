\chapter{Hardware a prostředí}

\section{Inkplate 10}
Středem celého projektu je panel Inkplate 10 od chorvatské společnosti Soldered Electronics. Ta jej uvedla na začátku roku 2021, ještě pod svým tehdejším jménem e-radionica \cite{solderedelectronicsUs}, na crowdfundingové platformě Crowd Supply \cite{zovkoOurCampaignLive2021}.
Kampaň měla za cíl vybrat 12 900 \$. To se jim podařilo vybrat za méně než 24 hodin \cite{crowdsupplyna[@crowd_supply]NewCampaignLaunched2021} a ve výsledku e-radionica vybrala přes 178 tisíc dolarů \cite{solderedelectronicsInkplate102021}.

Panel disponuje 9,7 palcovým e-ink displejem o rozlišení 1200 na 825 pixelů, USB-C portem, třemi kapacitními tlačítky, a integrovaným čipem ESP32.

Oproti jiným panelům, jako jsou např. Waveshare EPD a Adafrui EPD, lze Inkplate 10 používat bez dodatečného hardwaru. Kromě samotného e-ink displeje je na jeho desce zabudovaný i modul mikrokontroleru ESP32-WROVER. Díky němu lze do Inkplatu snadno nahrát vlastní firmware a zobrazit například fotky ze zabudované čtečky MicroSD karet či z internetu přes zabudovanou Wi-Fi anténu \cite{solderedelectronicsInkplate102021}.

\section{Možnosti zobrazení a omezení}

Displej, funguje na technologii elektronického inkoustu a pochází z vyřazených zařízení několika výrobců. Mezi hlavní zdroje patří čtečky Amazon Kindle \cite{solderedelectronicsInkplate102021}. Obraz na panelu zůstává a je vidět i po odebrání napájení. Díky tomu je při málo frekventovaných aktualizacích vysoce úsporný. V případě mikrokontroleru stačí systém probudit z hlubokého spánku např. jen jednou za minutu, zjistit požadovaný stav, překreslit displej a na celou následující minutu se znovu uspat. Displej zůstane nadále čitelný \cite{heikenfeldReviewPaperCritical2011}.

Každý pixel přitom může nabývat až 8 odstínů šedi, tedy 3 bity na jeden pixel. Aktualizace obrazu v tomto režimu trvá, dle výrobce, 1,81 sekundy.
V černobílém režimu (režim 1 bit na pixel) podporuje i částečné překreslení, které zvládá za 0,62 sekundy \cite{solderedelectronicsInkplate102021}. Při mém testování byly tyto časy ale o něco delší.

Důvodem pro relativně dlouhý čas pro překreslení panelu je technologie e-ink, tedy elektronický inkoust. Technologie (černo-bílého e-inkoustu) funguje na bázi fyzických miniaturních mikrokapslí, zhruba o průměru lidského vlasu. Každá kapsle obsahuje kladně nabité bílé částice a záporně nabité černé částice zachycené v čiré kapalině. Podle polarity elektrického pole v kapsli se černé a bílé částice rozdělí ke korespondujícím elektrodám u povrchu, čímž se daná kapsle zbarví do bíla či černa \cite{einkholdingsinc.ElectronicInkInk}. V případě panelů podporujících odstíny šedi, nejsou částice přesunuty plně k povrchu, ale jen částečně tak, aby poměr bílých a černých částic u povrchu na pohled připomínal cílený odstín šedi \cite{adafruitindustriesTHINKINKHow2020}. Jsou k tomu použity velmi přesné střídavé náboje. Pomalé vykreslování je způsobené relativně vysokou dobou přesunu částic a jejich ustálením. Navíc přesun částic není plně spolehlivý a při vykreslování je běžné, že některé skupiny částic zůstanou nepřesunuté, e-ink displeje bývá nutno relativně často resetovat cyklickým překreslením černého a bílého stavu \cite{heikenfeldReviewPaperCritical2011}.

\section{Programování Inkplate}

Pro účely programování nabízí Inkplate \cite{solderedelectronicsInkplate102021}\cite{solderedelectronicsInkplateFeaturesInkplate}:
\begin{itemize}
    \item dvoujádrový procesor s frekvencí až 240 MHZ
    \item 8 MB paměti RAM
    \item 4 MB paměti flash
    \item USB-C port
    \item integrovanou podporu Wi-Fi a Bluetooth 4.0
    \item integrovaný USB-UART konvertor
    \item integrovaný časovač (RTC)
    \item volně přístupné GPIO piny s nativní podporou protokolů I\textsuperscript{2}C, SPI a easyC/Qwiic
\end{itemize}

\subsection{ESP32}
Zabudovaný mikrokontroler je založený na čipu ze série ESP32 od čínské společnosti Espressif Systems \cite{espressifEspressifEspressifSystems}. Jedná se o nástupce čipu ESP8266 z roku 2014 uvedený o dva roky později v září 2016 \cite{espressifEspressifAnnouncesLaunch}. Tyto čipy jsou známé svou relativně nízkou cenou, integrovanou konektivitou, výkonem a otevřeností svého firmwaru. Staví se mezi nejčastěji používané mikrokontrolery na světě\cite{espressifEspressifLeadsIoT}\cite{espressifEspressifEspressifSystems}.

Je navržen pro mobilní zařízení, průmyslovou automatizaci a internet věcí. Pro účely omezení spotřeby energie nabízí několik režimů spánku, dynamickým škálováním výkonu a relativně nízkou aktivní spotřebu hlavního procesoru. V režimu Hlubokého spánku jeho spotřeba padne až na $10 \mu A$ i při zachování napájení RTC paměti. Pro aplikace kritické na bezpečnost nabídne hardwarově akcelerované kryptografické šifry, Secure Boot (ověření podpisu zaváděného firmwaru) a obecné šifrování flash paměti \cite{esp32_datasheet}. 

Všechny čipy ESP32 nabízí\cite{espressifProductOverviewESP32}:
\begin{itemize}
    \item Wi-Fi (2,4 GHz)
    \item Bluetooth 4.0
    \item dvoujádrový 32-bitový procesor Tensilica Xtensa LX6 operujiící až na 240 MHz
    \item úsporný koprocesor
\end{itemize}

\subsection{ESP-IDF}
Oficiální vývojové prostředí pro čipy řady ESP. Bylo vyvíjeno otevřeně pod open-source licensí Apache License 2.0 od uvedení ESP32 \cite{grokhotkovInitialPublicVersion2016}. ESP-IDF je založeno na implementaci operačního systému FreeRTOS, ale nabízí i ucházející podporu API kompatibilních se standardem POSIX. Pro příklad lze pro tvorbu vláken používat jak nativní \verb|xTaskCreate| z FreeRTOS tak i posixové \verb|pthread_create| \cite{espressifPOSIXThreadsSupport}. Zdrojový kód projektů závisejících na ESP-IDF se díky tomu téměř neliší od projektů pro běžné posixové operační systémy.

Od verze v2.0 podporuje projekty využívající C++ a STL \cite{grokhotkovReleaseESPIDFRelease} a ve své aktuální verzi v5.2 podporuje kompiluje se standardem C++23. S omezeními zvládá podporu C++ vláken, mutex, výjimek i RTTI \cite{espressifSupportESP32ESPIDF}. Vlákna a mutex jsou postavené nad interní implementací \verb|pthread.h| \cite{espressifPOSIXThreadsSupport}.

Kromě standardní knihovny C/C++ obsahuje SDK funkce pro přístup k hardwaru čipů ESP, jako je Wi-Fi, Bluetooth, persistentní paměť, JTAG debugging, HAL a LL abstrakce, zabezpečení bootloaderu a Over-The-Air aktualizace \cite{espressifAPIGuidesESP32}.

\subsection{Oficiální knihovny}
Soldered Electronics pro své panely vyvíjí a udržuje knihovny pro dvě vývojové platformy. Arduino a MicroPython. Nabízejí jednotné API pro práci s jakýmkoliv panelem od této společnosti. Obě pokrývají všechny základní potřeby pro interakci s displejem, přístup k hardwaru specifického pro jednotlivé panely a řadu pomocných funkcí např. pro dekódování obrázků či připojení k síti Wi-Fi.

\begin{table}[h]
\centering
\caption{Srovnán aktivity repozitářů knihoven pro vývoj pro panely Inkplate (větev master, 02.04.2024)}
\label{tab:inkplate-library-comparison}
\begin{tabular}{llll}
\multicolumn{1}{c}{}
    & Inkplate Arduino\cite{SolderedElectronicsInkplateArduinolibrary2024}
    & Inkplate MicroPython\cite{SolderedElectronicsInkplatemicropythonMicropython}
    & Inkplate ESP-IDF\cite{turcotteTurgu1ESPIDFInkPlate2024} \\ \hline
první commit             & 15.12.2019 & 17.08.2020  & 23.12.2019 \\
poslední commit          & 01.04.2024 & 19.02.2024  & 09.02.2023 \\
počet commitů            & 1330       & 123         & 162        \\
ot./uz. issues           & 3/140      & 5/15        & 5/8        \\ \hline
\end{tabular}
\end{table}

\subsubsection{Inkplate Arduino}
Platforma Arduino se skládá z mikrokontrolerů, programovacího jazyku, vývojového prostředí (IDE) a ekosystému Arduino knihoven. Arduino vzniklo jako učební pomůcka na Ivrea Interaction Design Institute v Milánu. Mělo za cíl vytvořit prostředí a vývojové desky, se kterými mohou pracovat i studenti bez podrobných znalostí elektroniky a programování \cite{WhatArduino}. Z čistě učební pomůcky se postupem času stala univerzální vývojová platforma sjednocující vývoj pro mikrokontrolery různých výrobců, tak i vlastní mikrokontrolery Arduino používané v enterprise IoT zařízeních \cite{Arduino}. Na softwarové platformě Arduino lze od roku 2018 vyvíjet i pro mikrokontrolery série ESP32 pomocí oficiální implementace Arduino Core \cite{ReleasesEspressifArduinoesp32}. Ta je vrstvou integrující softwarová API Arduino nad ESP-IDF. Lze tak nejen používat Arduino jazyk pro vývoj na ESP32, ale i integrovat knihovny určené pro Arduino v ESP-IDF projektu \cite{espressifsystemsArduinoESPIDFComponent}.

Na této kompatibilní vrstvě je postavena knihovna Inkplate Arduino. Pro Soldered je tato knihovna pro vlajkovou lodí. Mezi zmiňovanými knihovnami v této kapitole čítá její repozitář největšímu množství příspěvků a je mezi nimi též nejstarší \ref{tab:inkplate-library-comparison}.

\begin{lstlisting}[label=src:arduino-hello-world,language=C++,caption={Ukázka programu pro vykreslení kružnice a Hello World řetězce pomocí Arduino knihovny}]
#include "Inkplate.h"

// vytvoření instance ovladače
Inkplate display(INKPLATE_1BIT);

void setup() {
    // inicializace ovladače
    display.begin();
    // vyčištění displeje a bufferu
    display.clearDisplay();
    display.display();
    // vykreslení kružnice
    display.drawCircle(600, 412, 75, BLACK)
    // vykreslení textu
    display.setTextColor(BLACK, WHITE);
    display.setCursor(587, 550);
    display.print("Hello, world!");
    // částečné překreslení obrazovky
    display.partialUpdate();
}
\end{lstlisting}

\subsubsection{Inkplate MicroPython}
Druhá knihovna je určena pro firmware postavený na projektu MicroPython. Jedná se kompilátor a běhové prostředí implementující programovací jazyk Python 3 optimalizované pro běh na mikrokontrolerech a v omezených prostředí. Snaží se být co nejvíce kompatbilní s běžnými desktopovými verzemi Pythonu. Nabízí tak interaktivní příkazovou řádku, správu výjimek, výpočty s libovolnou přesností, věstavěný souborový systém \cite{MicroPythonPythonMicrocontrollers}.

V čem se MicroPython razantně liší od od Arduina, je proces vývoje. V prostředí Arduino (a ESP-IDF) je vyvíjená aplikace kompilována na běžném počítači spolu s celým firmwarem určeného pro mikrokontroler. Výsledný obraz sice bývá optimálnější na využití prostředků mikrokontroleru, na druhou stranu tento přístup s sebou nese náročnější proces při testování aplikace a komplexnější soubory projektu jako celku. Po instalaci běhového prostředí MicroPythonu na ESP32, lze desku připojit k počítači, kde se dá připojit k interaktivní Python konzoli běžící na mikrokontroleru a vyvíjené programy upravovat a spouštět z emulovaného flash úložiště \cite{MicroPythonPythonMicrocontrollers}. Na některých deskách lze interní filesystem připojit k systému jako běznou USB klíčenku a upravovat uložené python skripty napřímo. Není potřeba nic kompilovat, nic flashovat, ani mít v systému speciální vývojové nástroje \cite{WorkingFilesystemsMicroPython}. V případě, že mikrokontroler nepodporuje režim USB MSC, je pro kopírování souborů na desku nutno použít nástroj \verb|pytboard.py|, který však lze verzovat uvnitř projektu a je obsažen i v projektu Inkplate knihovny \cite{PyboardPyTool}.

Protože vyvíjená aplikace nemusí projít kompilací před jejím nahráním na desku, je kompilátor obsažen v MicroPython firmwaru. To má vliv na použitelnou velikost flash paměti pro vyvíjený program. Stejně tak tím, že je nutné skripty při načítání zpracovat a zkompilovat do Python bajtkódu, rychlost aplikací založených na běhovém prostředí MicroPython bývá citelně nižší než jejich kompilovaná alternativa v C/C++, jako je například ESP-IDF v případě ESP32 \cite{plauskaPerformanceEvaluationMicroPython2022}.

\begin{lstlisting}[label=src:micropython-hello-world,language=Python,caption={Ukázka programu pro vykreslení kružnice a Hello World řetězce pomocí MicroPython knihovny}]
from inkplate10 import Inkplate

// vytvoření instance ovladače
display = Inkplate(Inkplate.INKPLATE_1BIT)

if __name__ == "__main__":
    // inicializace ovladače
    display.begin()
    // vyčištění displeje a bufferu
    display.clearDisplay()
    display.display()
    // vykreslení kružnice
    display.drawCircle(600, 412, 75, display.BLACK)
    // vykreslení textu
    display.printText(587, 550, "Hello, world!");
    // částečné překreslení obrazovky
    display.partialUpdate()
}
\end{lstlisting}

\subsection{Komunitní knihovny}
\subsubsection{Inkplate ESP-IDF}

Třetí oficiálně doporučenou knihovnou je Inkplate ESP-IDF. Ta však není vyvíjena přímo společností Soldered Electronics, ale vývojářem Guyem Turcottem. Jde o port původní Inkplate Arduino knihovny do prostředí ESP-IDF a platformy PlatformIO. Turcott původní knihovnu z velké části přepsal tak, aby závisela jen na nativní standardní knihovně ESP-IDF, používala C++ syntaxi a bylo minimalizováno využití maker \cite{ESPIDFInkPlateREADMEMd}.

Port vznikl pro účely Turcottova projeku čtečky elektronických knih. Čtečka byla určena pro osobou mu blízkou, která po úrazu utrpěla poranění míchy a zůstala upoutána na lůžku jen s omezeným pohybem končetin. Turcott jí symbolicky projekt věnoval, po její smrti na Covid-19, koncem roku 2020 \cite{turcotteTurgu1EPubInkPlate2024}.

Na této knihovně mám částečný podíl i já, protože jsem si ji zvolil jako výchozí knihovnu pro interakci se zapůjčeným panel v rámci této práce a do upstreamu se dostalo pár změn, které jsem připravil ve svém forku. 

\paragraph{PlatformIO} Původní verze Inkplate ESP-IDF byla připravena pro integraci v platformě PlatformIO. Ta slouží jako nástroj usnadňující práci s vývojovými prostředími embedded systémů. Sjednocuje tak příkazy pro inicializaci správné verze prostředí, konfiguraci parametrů desky, sestavení, flash apod \cite{WhatPlatformIOPlatformIO}. V době, kdy jsem ale projekt zakládal, nepodporovalo PlatformIO IDF verze 5.0 a nefungovala mi její integrace v prostředí editoru CLion. Naštěstí bylo potřeba jen vytvořit soubor \verb|CMakeLists.txt| registrující knihovnu jako komponentu ESP-IDF a zdokumentovat ruční instalaci knihovny do existujících ESP-IDF projektů.

\paragraph{ESP-IDF v5.0} Ve verzi 5.0 ESP-IDF přineslo řadu nových funkcí a zároveň i množství zpěně nekompatibilních změn. Knihovna s touto verzí nebyla kompatiblní. IDF změnilo verzi GCC a s ní verzi cílového standardu C++. Ve verzi C++20 byla například odebrána podpora smíšeného bitového přiřazení volatilních typů \cite{jtcDedeprecatingVolatileCompound}. V rámci knihovny ESP-IDF se zase změnily názvy některých konstant\cite{PeripheralsESP32S2ESPIDF} a byl odebrána stará verz ovladače SD karty \cite{StorageESP32S2ESPIDF}, kterou Inkplate ESP-IDF interně používal. Jednotlivé rozbité části jsem opravil a byly připojeny do větve \verb|idf-v5.0-support| upstream repozitáře \cite{WIPInitialSupport}. 

\paragraph{Chyba certifikátu} Knihovna podporovala načítání obrázků ze vzdálené http adresy, ale padala při snaze o načtení https adres. Bylo nutné přiřadit do interního síťového klienta balík s certifikáty \cite{MenuconfigVariantSelection}.

\paragraph{Znovunavázání Wi-Fi spojení} Inkplate ESP-IDF obsahuje pomocnou funkci \verb|Display::joinAP()|, pro usnadnějí připojení desky k síti. Při testování jsem ale, bohužel, zjistil, že se při nepodařeném spojení \verb|joinAp()| již znovu nepřipojí. Na vině bylo chybné resetování adaptéru ESP-NETIF a kontroly připojeného stavu \cite{MenuconfigVariantSelection}


\begin{lstlisting}[label=src:esp-idf-hello-world,language=C++,caption={Ukázka programu pro vykreslení kružnice a Hello World řetězce pomocí ESP-IDF knihovny}]
#include "inkplate.hpp"

// vytvoření instance ovladače
Inkplate display(DisplayMode::INKPLATE_1BIT);

extern "C" {
    void app_main() {
        // inicializace ovladače
        display.begin();
        // vyčištění displeje a bufferu
        display.clearDisplay();
        display.display();
        // vykreslení kružnice
        display.drawCircle(600, 412, 75, BLACK)
        // vykreslení textu
        display.setTextColor(BLACK, WHITE);
        display.setCursor(587, 550);
        display.print("Hello, world!");
        // částečné překreslení obrazovky
        display.partialUpdate();
    }
}
\end{lstlisting}

\subsection{Režim periferie}
Mimo samostatný režim, při kterém aplikace ovládající displej běží na zabudovaném mikrokontroleru, nabízí Inkplate i režim periferie. V takovém režimu panel naslouchá na rozhraní UART na textové příkazy, které převádí na volání funkcí oficiální knihovny. Na nových panelech od Soldered bývá předinstalován jako výchozí firmware. Protože se ale jedná o firmware běžící na ESP32 a ne na samotném displeji, je distribuovaný jako ukázkový projekt v repozitáři oficiálního Arduino ovladače a je na vývojáři, aby si jej v případě potřeby zkompiloval a nahrál do zařízení \cite{SolderedElectronicsInkplatePeripheralModeRaspberryPiExample2023}.

Práce s displejem je podobná ostatním zmíněným knihovnám a podporované funkce se liší primárně nutností korektně formátovat odesílané příkazy \cite{solderedelectronicsInkplatePeripheralMode}.

\begin{lstlisting}[label=src:PeripheralHelloWorld,caption={Ukázka příkazů pro vykreslení kružnice a Hello World řetězce v režimu periferie}]
// -- řádky odděleny /r/n ---
// vyčištění displeje a bufferu
#K(1)*
#L(1)*
// vykreslení kružnice
#5(600,412,075,01)*
// vykreslení textu (řetězec je nutné převést do hexadecimálního tvaru)
#E(587,550)*
#C("48656c6c6f2c20776f726c6421")*
// částečné překreslení obrazovky
#M(000,1199,824)*
\end{lstlisting}
