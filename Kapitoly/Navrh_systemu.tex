\chapter{Návrh systému}

Od počátku jsem navrhoval systém tak, aby panel byl co možná nejhloupější. Panel by měl jen naslouchat příkazům serveru a ty provádět. To zahrnuje příjem zobrazených dat jen ve formě bitmapových obrázků. Všechna obrazová data jsou díky tomu uniformní a ze strany serveru je možné pevně nastavit každý jednotlivý pixel.

Dále jsem věděl, že systém bude mít vždy alespoň tři části. Panel, který bude naslouchat příkazům serveru. Server, který bude udržovat nějakým způsobem komunikaci s panely. A pak klient, který bude skrze server panely ovládat. Primárně jsem nechtěl spojovat server s klientem v jeden monolit, aby bylo snazší projekt v budoucnu rozšířit například o mobilní nebo terminálovou aplikaci.

\section{Nedokončené návrhy a implementace}
Na projektu jsem pracoval průběžně v rámci celých tří let. V průběhu této doby systém prošel mnoha změnami a několika plnými přepisy. Zde jsou vybrány tři verze, které se dostaly do funkčních fází.

\subsection{Hamilton}
Pojmenován po softwarové inženýrce Margaret Hamilton, má první seriozní implementace systému spočívala v komunikaci panelu a serveru přes protokol WebSocket. Zprávy byly ve formátu JSON a server sloužil jen jako prostředník mezi klientem a panely. Server udržoval s panely udržoval spojení a klient mohl přes HTTP API serveru poslat panelům zprávu.

\begin{figure}[h]
    \centering
    \begin{tikzpicture}[node distance=2cm]
        \node (server) [rect]  {server};
        \node (panel) [rect, below left of=server] {panel};
        \node (klient) [rect, below right of=server] {klient};
        \draw[->] (panel) -- (server) node[midway, left] {websocket};
        \draw[->] (klient) -- (server) node[midway, right] {http};
    \end{tikzpicture}
    \caption{Diagram architektury první verze systému}
    \label{fig:first-system-architecture}
\end{figure}

\subsubsection{Server}
Server měl jen čtyři endpointy a byl založen technologii Node.js a frameworku fastify.

\paragraph*{/:panelId} Sloužil k připojení panelu k serveru pomocí WebSocket spojení. Podle atributu \lstinline{panelId} byl panel identifikován a přidán do seznamu běžících panelů. Udržována bylo jeho id, ip adresa a status.
\paragraph*{/panels} Vracel seznam připojených panelů k serveru.
\paragraph*{/images/:imageName} Statické soubory obrázků.
\paragraph*{/panels/broadcast} Očekával zprávu ve formátu JSON, kterou poté přeposlal panelům. Pokud zpráva byla typu DRAW\_IMAGE\_FROM\_URL, server si z adresy vyrobil hash, obrázek stáhl, upravil jeho velikost, převedl do JPEG formátu v odstínech šedi a uložil do lokální složky pod vygenerovaným hashem. Panelům pak zprávu přeposlal s adresou převedeného obrázku hostovaného na adrese serveru.

\subsubsection{Panel}
Firmware panelu byl založen na prostředí ESP-IDF a knihovny Inkplate ESP-IDF. Podporoval osm druhů příkazů a upozornění na stisknutí kapacitních tlačítek. Firmware byl rozdělen do pěti souběžných vláken, implementovaných pomocí FreeRTOS tasků. Vlákna mezi sebou komunikovaly přes sdílenou smyčku událostí.

\subsection{Cron}

\section{Konečný návrh}

\subsection{Scheduler}
\subsection{Hoster}
\subsection{Renderer}
\subsection{Compressor}
\subsection{Grouper}
\subsection{Panel}
\subsection{Facade}