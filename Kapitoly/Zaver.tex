\chapter{Závěr}

V rámci této diplomové práce byl vyvinut systém pro správu informačního panelu s využitím modulu e-ink displeje Inkplate 10. Cílem bylo vytvořit program pro e-ink displej, který umožní zobrazování různých informací, jako je rozvrh místnosti, obrázky, jednoduché animace a další, a to prostřednictvím spojení s nadřazeným systémem přes síť Wi-Fi.

V rámci práce je stručně popsán použitý hardware. Dále je popsán stav oficiálně dostupných knihoven podporujících panely Inkplate. Při vývoji bylo zjištěno, že zvolená knihovna Inkplate ESP-IDF obsahuje určité nedostatky. Byl vytvořen fork této knihovny a byl upraven tak, aby ji bylo možné dále používat.

Byl vytvořen program, který umožňuje modulu e-ink displeje připojení k nadřazenému systému přes Wi-Fi s využitím MQTT protokolu. Program je schopen zobrazovat bitmapové obrázky a pomocí MQTT zpráv jej lze konfigurovat.

Pro účely uživatelského ovládání byla vytvořena sada serverových služeb a webové rozhraní. Systém umožňuje centrální správu více e-ink displejů a definování obsahu, který se na nich zobrazuje. Tento systém je přístupný skrze webové rozhraní, které podporuje autentizaci a umožňuje uživatelům přizpůsobit zobrazení na konkrétních displejích.

Systém byl nasazen na panelech u přednáškových místností EC1, EC2 a EC3. Tyto panely byly v nepřetržitém provozu po dobu 6 měsíců a většinu této doby zobrazovaly informace o počasí, čase a novinkách školy s aktualizací každou minutu.

V rámci testování se ukázaly nedostatky v efektivitě některých služeb, zejména služby Renderer, zodpovědné za zpracování webových dokumentů pro zobrazení na e-ink panelech. Tato část systému by měla být v budoucnu optimalizována tak, aby byla zajištěna kompatibilita s komplexními webovými dokumenty a včasného zobrazení obsahu. Případně musí být tato služba provozována na výkonnějším serveru, než který jsem měl k dispozici.

Osobně plánuji pokračovat ve vývoji systému i nadále. Panel Inkplate 10 jsem si pořídil pro soukromé účely a chci jej využít jako obrazovku pro zobrazení dat služby Home Assistant\cite{HomeassistantCoreHouse_with_garden}.

\newpage
Projekt je vyvíjen otevřeně na platformě GitHub pod organizací vsb-eink.
\begin{table}[h]
    \begin{tabular}{ll}
        Monorepozitář serveru + aplikace & \url{https://github.com/vsb-eink/vsb-eink-services} \\
        Firmware panelu & \url{https://github.com/vsb-eink/vsb-eink-panel} \\
    \end{tabular}
\end{table}