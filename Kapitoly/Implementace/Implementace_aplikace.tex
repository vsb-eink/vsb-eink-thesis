\section{Implementace aplikace}
Pro uživatelsky přívětivé ovládání systému bez potřeby přímého přístupu k serveru byla vyvinuta základní webová aplikace ve stylu SPA, tedy Single Page Application. Jde o druh webové aplikace, jejíž základ je popsán jen jediným webovým dokumentem a veškerý zbytek dat aplikace je načítán aplikací samotnou bez potřeby opakovaného načtení dokumentu\cite{SPASinglepageApplication2023}. Samotná aplikace je distribuována jako sada statických souborů a lze ji tak hostovat kdekoliv, včetně lokálního serveru. Dynamická data získává aplikace skrze REST API služby Facade, podobně jako jakýkoliv jiný samostatný klient. Její balíček je nazván \lstinline|dashboard-vue| a zastupuje službu \textbf{Dashboard}.

Za základní framework, na kterém je aplikace postavena, bylo zvoleno Vue 3, které svou syntaxí připomíná framework Svelte, který je již používaný ve školním systému Kelvin \cite{SPASinglepageApplication2023}. Pro zjednodušení tvorby uživatelského rozhraní jsem použil komponentový framework Quasar\cite{QuasarframeworkQuasar2024}. Ten jsem zvolil po srovnání s jinými dostupnými Vue frameworky. Quasar jako jediný nabídl komponentu pro interakci se stromovitými daty, nevyžadoval žádné stylování ze strany vývojáře a měl dostatečně aktivní vývoj a komunitu. Jmenovitě jsem mimo Quasar evaluoval i Shadcn\cite{RadixvueRadixvue2024}, Vuetify\cite{VuetifyjsVuetify2024}, PrimeVue\cite{PrimefacesPrimevueNext}, Naive UI\cite{TusenaiNaiveui2024} a Radix Vue\cite{RadixvueShadcnvue2024}.

\subsection{Použité nástroje}
\begin{itemize}
    \item editor: WebStorm
    \item jazyk: TypeScript
    \item klient API: openapi-generator\cite{OpenAPIToolsOpenapigeneratorOpenAPI}
    \item knihovny:
        \begin{itemize}
            \item Reaktivita --> Vue\cite{VuejsCore2024}
            \item UI komponenty --> Quasar\cite{QuasarframeworkQuasar2024}
        \end{itemize}
\end{itemize}

\subsection{Adresářová struktura projektu}
Kořen složky aplikace je uložen v repozitáři služeb v cestě \lstinline|/apps/dashboard-vue|. Seznam souborů níže popisuje obsah složky aplikace \lstinline|src/|.

\begin{itemize}
    \item assets/
        \begin{itemize}
            \item obsahuje soubory CSS stylů aplikace a případně soubory jako obrázky, fonty
        \end{itemize}
    \item components/
        \begin{itemize}
            \item obsahuje ,,hloupé'' komponenty, které nemají žádnou logiku a slouží jen k zobrazení dat
        \end{itemize}
    \item composables/
        \begin{itemize}
            \item obsahuje tzv. ,,composable'' funkce, které slouží k sdílení logiky mezi komponentami
        \end{itemize}
    \item layouts/
        \begin{itemize}
            \item obsahuje komponenty pro sdílení struktury stránek
        \end{itemize}
    \item router/
        \begin{itemize}
            \item obsahuje konfiguraci routeru aplikace a definice cest
        \end{itemize}
    \item services/
        \begin{itemize}
            \item obsahuje služby pro komunikaci s API
        \end{itemize}
    \item utils/
        \begin{itemize}
            \item obsahuje různé pomocné funkce
        \end{itemize}
    \item views/
        \begin{itemize}
            \item obsahuje komponenty pro jednotlivé obrazovky aplikace
        \end{itemize}
    \item App.vue
        \begin{itemize}
            \item hlavní komponenta aplikace
            \item obsahuje router-view a inicializaci uživatelova stavu
        \end{itemize}
    \item environment.ts
        \begin{itemize}
            \item obsahuje konfiguraci aplikace načtenou z proměnných prostředí
        \end{itemize}
    \item main.ts
        \begin{itemize}
            \item vstupní bod aplikace
            \item inicializace Vue aplikace
            \item inicializace frameworku Quasar
            \item inicializace knihovny Pinia\cite{VuejsPinia2024}
        \end{itemize}
\end{itemize}

\subsection{API klient}
Kód klienta k Facade API byl vygenerován nástrojem openapi-generator organizace OpenAPI Tools. Specificky intergrovaný generátor typescript-axios. Ten dostává na vstupu OpenAPI specifikaci a generuje soubory typově bezpečného klienta s podporou autentizace, variabilní adresou serveru, nahrávání souborů a metodami pojmenovanými podle atributu \lstinline|operationId| uvedeným ve specifikaci pro každou cestu.

\subsubsection{Úpravy klienta}
Kód vygenerovaného klienta bylo potřeba upravit z důvodu limitace OpenAPI popisu adres uživatelských souborů. Nepodporuje totiž adresní parametry obsahující lomítka, nelze je tedy s její pomocí validně popsat. Vygenerovaný klient parametry adresy, dle specifikace, escapuje funkcí \lstinline|encodeURIComponent|. Pro možnost zachování používání, jinak nezměněného, klienta, jsem napsal skript, který po generaci vyhledá a přepíše instance kódu escapující parametry \lstinline|path| z kódující funkce jen na běžné převedení na řetězec. Příloha \ref{src:input-getmetadata-openapi} ukazuje popis cesty \lstinline|GET /hosted/core/files/{path}| pomocí OpenAPI dokumentu. V příloze \ref{src:output-getmetadata-openapi} je potom vidět vygenerovaná funkce \lstinline|getContentMetadata|.

\begin{lstlisting}[label=src:facade-client-init,caption={Inicializace Facade API klienta}]
const axiosInstance = axios.create({
	baseURL: FACADE_URL,
});
const apiConfig = new Configuration({
	accessToken: () => getToken() || '',
});
export const api = {
	auth: AuthApiFactory(apiConfig, undefined, axiosInstance),
	admin: AdminApiFactory(apiConfig, undefined, axiosInstance),
	hosted: HostedApiFactory(apiConfig, undefined, axiosInstance),
	panels: PanelsApiFactory(apiConfig, undefined, axiosInstance),
	users: UsersApiFactory(apiConfig, undefined, axiosInstance),
	schedule: ScheduleApiFactory(apiConfig, undefined, axiosInstance),
};
\end{lstlisting}

Při inicializaci je vytvořena instance HTTP klienta axios\cite{AxiosAxios2024}, které je předána kořenová adresa služby Facade. Autentizační token je generovaným klientem načítán při každém požadavku funkcí \lstinline|accessToken()|, stačí ji tedy naimplementovat vlastní funkcí, která token načte z lokálního úložiště. Klient má pak vždy aktuální token. Ukázka kódu \ref{src:facade-client-init} zachycuje inicializaci klienta.

Klient je umístěn ve vlastním balíčku \lstinline|facade-api-client| a je samostatně importovatelný.

\subsection{Ovládání aplikace}
Obrazovky aplikace pevně kopírují strukturu a funkcionalitu Facade API. Uživatel má možnost procházet seznamy uživatelů, panelů, úkolů (obrázek \ref{fig:responsive-schedules}) a hostovaných souborů. Po výběru entity je uživatel přesměrován na obrazovku s detaily entity, kde je entita modifikovatelná. Po pozměnění dat entity je uživateli nabídnuto uložení změn a změny jsou odeslány na server.

\begin{figure}[h]
    \centering
    \subfloat[Obrazovka plánovaných úkolů (Desktop)\label{fig:eink-schedules-view}]{
        \includegraphics[width=0.70\textwidth]{Obrazky/aplikace/eink.a1314.cz_schedules.png}
    }
    \subfloat[Obrazovka plánovaných úkolů (Pixel 7)\label{fig:eink-schedules-pixel7-view}]{
        \includegraphics[width=0.28\textwidth]{Obrazky/aplikace/eink.a1314.cz_schedules_pixel7.png}
    }
    \caption{Srovnání obrazovek plánovaných úkolů na různých zařízeních}
	\label{fig:responsive-schedules}
\end{figure}
