\chapter{Úvod}

Elektronické inkoustové panely, známé také jako e-ink displeje, představují revoluční technologii zobrazování informací, která nachází uplatnění v širokém spektru aplikací od čteček elektronických knih až po informační tabule. Fakulta elektrotechniky a informatiky VŠB-TUO se rozhodla využít tuto technologii k zefektivnění komunikace uvnitř budovy. Cílem tohoto projektu je vytvořit systém, který umožní dynamické zobrazování obsahu na e-ink displejích prostřednictvím WiFi, čímž se sníží potřeba manuální aktualizace a zvýší se efektivita správy fakultních prostor. Implementace tohoto systému umožní flexibilní a okamžité šíření informací spjatých s vybranými místy fakulty.

Prvním z cílových míst jsou zasedací místnosti, u kterých je běžné, že se průběžně mění rozpis jejich rezervací. Bez denně aktualizované tabulky rozpisu nelze jednoduše zjistit, jaká schůzka se zde právě koná a kdy bude místnost opět volná. S elektronickým panelem lze rozpis automaticky aktualizovat a mít tak vždy aktuální přehled o rezervaci místnosti.

Přednáškové místnosti jsou dalším z míst, kde je potřebné mít možnost dynamického rozvrhu. Z technických důvodů mohou být přednášky přesunuty do jiné místnosti nebo se může konat mimořádná přednáška mimo pravidelný rozvrh. Panely ale nemusí sloužit jen pro účely rozvrhu. Mohou ukazovat čas, pokyny od pedagogů, promo obsah, fotky nebo akutní varování.

Práce je rozdělena do pěti základních kapitol. První kapitola popisuje hardware a prostředí, pro které je systém určen. Zaměřena je především na popis samotných panelů, na kterých je tato práce postavena. Je zmíněna jejich softwarová a hardwarová architektura, možnosti, které nabízejí a limitace, se kterými bylo nutno pracovat při vývoji a návrhu. Zahrnuta je také analýza technologie elektronického inkoustu, jeho výhod, omezení v kontextu tohoto projektu.

Druhá kapitola se zaměřuje na systém jako celek. Prochází historii návrhu systému a zmiňuje slepé uličky, které tvarovaly podobu konečného systému. Finálem kapitoly je popis stávající architektury systému. Slouží jako základ ke snadnějšímu porozumění následujících kapitol o jednotlivých částech systému. Důraz je kladen na modulární design a flexibilitu systému, které umožňují snadnou adaptabilitu na měnící se požadavky a integraci nových funkcionalit.

Třetí kapitola se věnuje samotné implementaci jednotlivých částí navrženého systému. Každá její část obsahuje popis implementovaných API. Od návrhu po jejich implementaci. V první části kapitoly je popsána implementace firmwaru panelu. V další je popsán backend systému. Poslední část se zabývá implementací webové aplikace sloužící k autentizované interakci s jednotlivými komponentami. Rozbor každé části zahrnuje technické detaily a výzvy, se kterými jsem se setkal, a způsoby, jakými jsem se rozhodl výzvy řešit.

Čtvrtá kapitola popisuje, jak byl systém nasazen v produkčním prostředí a jaké technologie jsou použity pro jeho sestavení a údržbu. Tato část završuje celý proces vývoje.

V poslední kapitole jsou zmíněny charakteristiky systému z hlediska stability a spolehlivosti ve skutečném provozu.